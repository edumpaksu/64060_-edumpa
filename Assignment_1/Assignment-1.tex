% Options for packages loaded elsewhere
\PassOptionsToPackage{unicode}{hyperref}
\PassOptionsToPackage{hyphens}{url}
%
\documentclass[
]{article}
\usepackage{amsmath,amssymb}
\usepackage{iftex}
\ifPDFTeX
  \usepackage[T1]{fontenc}
  \usepackage[utf8]{inputenc}
  \usepackage{textcomp} % provide euro and other symbols
\else % if luatex or xetex
  \usepackage{unicode-math} % this also loads fontspec
  \defaultfontfeatures{Scale=MatchLowercase}
  \defaultfontfeatures[\rmfamily]{Ligatures=TeX,Scale=1}
\fi
\usepackage{lmodern}
\ifPDFTeX\else
  % xetex/luatex font selection
\fi
% Use upquote if available, for straight quotes in verbatim environments
\IfFileExists{upquote.sty}{\usepackage{upquote}}{}
\IfFileExists{microtype.sty}{% use microtype if available
  \usepackage[]{microtype}
  \UseMicrotypeSet[protrusion]{basicmath} % disable protrusion for tt fonts
}{}
\makeatletter
\@ifundefined{KOMAClassName}{% if non-KOMA class
  \IfFileExists{parskip.sty}{%
    \usepackage{parskip}
  }{% else
    \setlength{\parindent}{0pt}
    \setlength{\parskip}{6pt plus 2pt minus 1pt}}
}{% if KOMA class
  \KOMAoptions{parskip=half}}
\makeatother
\usepackage{xcolor}
\usepackage[margin=1in]{geometry}
\usepackage{color}
\usepackage{fancyvrb}
\newcommand{\VerbBar}{|}
\newcommand{\VERB}{\Verb[commandchars=\\\{\}]}
\DefineVerbatimEnvironment{Highlighting}{Verbatim}{commandchars=\\\{\}}
% Add ',fontsize=\small' for more characters per line
\usepackage{framed}
\definecolor{shadecolor}{RGB}{248,248,248}
\newenvironment{Shaded}{\begin{snugshade}}{\end{snugshade}}
\newcommand{\AlertTok}[1]{\textcolor[rgb]{0.94,0.16,0.16}{#1}}
\newcommand{\AnnotationTok}[1]{\textcolor[rgb]{0.56,0.35,0.01}{\textbf{\textit{#1}}}}
\newcommand{\AttributeTok}[1]{\textcolor[rgb]{0.13,0.29,0.53}{#1}}
\newcommand{\BaseNTok}[1]{\textcolor[rgb]{0.00,0.00,0.81}{#1}}
\newcommand{\BuiltInTok}[1]{#1}
\newcommand{\CharTok}[1]{\textcolor[rgb]{0.31,0.60,0.02}{#1}}
\newcommand{\CommentTok}[1]{\textcolor[rgb]{0.56,0.35,0.01}{\textit{#1}}}
\newcommand{\CommentVarTok}[1]{\textcolor[rgb]{0.56,0.35,0.01}{\textbf{\textit{#1}}}}
\newcommand{\ConstantTok}[1]{\textcolor[rgb]{0.56,0.35,0.01}{#1}}
\newcommand{\ControlFlowTok}[1]{\textcolor[rgb]{0.13,0.29,0.53}{\textbf{#1}}}
\newcommand{\DataTypeTok}[1]{\textcolor[rgb]{0.13,0.29,0.53}{#1}}
\newcommand{\DecValTok}[1]{\textcolor[rgb]{0.00,0.00,0.81}{#1}}
\newcommand{\DocumentationTok}[1]{\textcolor[rgb]{0.56,0.35,0.01}{\textbf{\textit{#1}}}}
\newcommand{\ErrorTok}[1]{\textcolor[rgb]{0.64,0.00,0.00}{\textbf{#1}}}
\newcommand{\ExtensionTok}[1]{#1}
\newcommand{\FloatTok}[1]{\textcolor[rgb]{0.00,0.00,0.81}{#1}}
\newcommand{\FunctionTok}[1]{\textcolor[rgb]{0.13,0.29,0.53}{\textbf{#1}}}
\newcommand{\ImportTok}[1]{#1}
\newcommand{\InformationTok}[1]{\textcolor[rgb]{0.56,0.35,0.01}{\textbf{\textit{#1}}}}
\newcommand{\KeywordTok}[1]{\textcolor[rgb]{0.13,0.29,0.53}{\textbf{#1}}}
\newcommand{\NormalTok}[1]{#1}
\newcommand{\OperatorTok}[1]{\textcolor[rgb]{0.81,0.36,0.00}{\textbf{#1}}}
\newcommand{\OtherTok}[1]{\textcolor[rgb]{0.56,0.35,0.01}{#1}}
\newcommand{\PreprocessorTok}[1]{\textcolor[rgb]{0.56,0.35,0.01}{\textit{#1}}}
\newcommand{\RegionMarkerTok}[1]{#1}
\newcommand{\SpecialCharTok}[1]{\textcolor[rgb]{0.81,0.36,0.00}{\textbf{#1}}}
\newcommand{\SpecialStringTok}[1]{\textcolor[rgb]{0.31,0.60,0.02}{#1}}
\newcommand{\StringTok}[1]{\textcolor[rgb]{0.31,0.60,0.02}{#1}}
\newcommand{\VariableTok}[1]{\textcolor[rgb]{0.00,0.00,0.00}{#1}}
\newcommand{\VerbatimStringTok}[1]{\textcolor[rgb]{0.31,0.60,0.02}{#1}}
\newcommand{\WarningTok}[1]{\textcolor[rgb]{0.56,0.35,0.01}{\textbf{\textit{#1}}}}
\usepackage{graphicx}
\makeatletter
\def\maxwidth{\ifdim\Gin@nat@width>\linewidth\linewidth\else\Gin@nat@width\fi}
\def\maxheight{\ifdim\Gin@nat@height>\textheight\textheight\else\Gin@nat@height\fi}
\makeatother
% Scale images if necessary, so that they will not overflow the page
% margins by default, and it is still possible to overwrite the defaults
% using explicit options in \includegraphics[width, height, ...]{}
\setkeys{Gin}{width=\maxwidth,height=\maxheight,keepaspectratio}
% Set default figure placement to htbp
\makeatletter
\def\fps@figure{htbp}
\makeatother
\setlength{\emergencystretch}{3em} % prevent overfull lines
\providecommand{\tightlist}{%
  \setlength{\itemsep}{0pt}\setlength{\parskip}{0pt}}
\setcounter{secnumdepth}{-\maxdimen} % remove section numbering
\ifLuaTeX
  \usepackage{selnolig}  % disable illegal ligatures
\fi
\IfFileExists{bookmark.sty}{\usepackage{bookmark}}{\usepackage{hyperref}}
\IfFileExists{xurl.sty}{\usepackage{xurl}}{} % add URL line breaks if available
\urlstyle{same}
\hypersetup{
  pdftitle={Assignment 1},
  pdfauthor={Eswar Dumpa},
  hidelinks,
  pdfcreator={LaTeX via pandoc}}

\title{Assignment 1}
\author{Eswar Dumpa}
\date{2024-02-04}

\begin{document}
\maketitle

\hypertarget{importing-dataset}{%
\subsection{Importing Dataset}\label{importing-dataset}}

\begin{Shaded}
\begin{Highlighting}[]
\NormalTok{diabetes }\OtherTok{\textless{}{-}} \FunctionTok{read.csv}\NormalTok{(}\StringTok{"diabetes.csv"}\NormalTok{)}
\FunctionTok{head}\NormalTok{(diabetes)}
\end{Highlighting}
\end{Shaded}

\begin{verbatim}
##   Pregnancies Glucose BloodPressure SkinThickness Insulin  BMI
## 1           6     148            72            35       0 33.6
## 2           1      85            66            29       0 26.6
## 3           8     183            64             0       0 23.3
## 4           1      89            66            23      94 28.1
## 5           0     137            40            35     168 43.1
## 6           5     116            74             0       0 25.6
##   DiabetesPedigreeFunction Age Outcome
## 1                    0.627  50       1
## 2                    0.351  31       0
## 3                    0.672  32       1
## 4                    0.167  21       0
## 5                    2.288  33       1
## 6                    0.201  30       0
\end{verbatim}

\#Importing data from kaggle dataset using the below link
\url{https://www.kaggle.com/datasets/akshaydattatraykhare/diabetes-dataset}

\hypertarget{descriptive-statistics}{%
\subsection{Descriptive Statistics}\label{descriptive-statistics}}

\begin{Shaded}
\begin{Highlighting}[]
\FunctionTok{summary}\NormalTok{(diabetes)}
\end{Highlighting}
\end{Shaded}

\begin{verbatim}
##   Pregnancies        Glucose      BloodPressure    SkinThickness  
##  Min.   : 0.000   Min.   :  0.0   Min.   :  0.00   Min.   : 0.00  
##  1st Qu.: 1.000   1st Qu.: 99.0   1st Qu.: 62.00   1st Qu.: 0.00  
##  Median : 3.000   Median :117.0   Median : 72.00   Median :23.00  
##  Mean   : 3.845   Mean   :120.9   Mean   : 69.11   Mean   :20.54  
##  3rd Qu.: 6.000   3rd Qu.:140.2   3rd Qu.: 80.00   3rd Qu.:32.00  
##  Max.   :17.000   Max.   :199.0   Max.   :122.00   Max.   :99.00  
##     Insulin           BMI        DiabetesPedigreeFunction      Age       
##  Min.   :  0.0   Min.   : 0.00   Min.   :0.0780           Min.   :21.00  
##  1st Qu.:  0.0   1st Qu.:27.30   1st Qu.:0.2437           1st Qu.:24.00  
##  Median : 30.5   Median :32.00   Median :0.3725           Median :29.00  
##  Mean   : 79.8   Mean   :31.99   Mean   :0.4719           Mean   :33.24  
##  3rd Qu.:127.2   3rd Qu.:36.60   3rd Qu.:0.6262           3rd Qu.:41.00  
##  Max.   :846.0   Max.   :67.10   Max.   :2.4200           Max.   :81.00  
##     Outcome     
##  Min.   :0.000  
##  1st Qu.:0.000  
##  Median :0.000  
##  Mean   :0.349  
##  3rd Qu.:1.000  
##  Max.   :1.000
\end{verbatim}

\#it displays the summary statistics of the ``diabetes'' Dataset.The
Dataset contains 768 observations.

\hypertarget{transform}{%
\subsection{Transform}\label{transform}}

\begin{Shaded}
\begin{Highlighting}[]
\FunctionTok{library}\NormalTok{(dplyr)}
\end{Highlighting}
\end{Shaded}

\begin{verbatim}
## 
## Attaching package: 'dplyr'
\end{verbatim}

\begin{verbatim}
## The following objects are masked from 'package:stats':
## 
##     filter, lag
\end{verbatim}

\begin{verbatim}
## The following objects are masked from 'package:base':
## 
##     intersect, setdiff, setequal, union
\end{verbatim}

\begin{Shaded}
\begin{Highlighting}[]
\NormalTok{diabetes\_2}\OtherTok{\textless{}{-}} \FunctionTok{mutate}\NormalTok{(diabetes,}\AttributeTok{Outcome=}\FunctionTok{case\_when}\NormalTok{(}
\NormalTok{  Outcome}\SpecialCharTok{==}\DecValTok{1} \SpecialCharTok{\textasciitilde{}} \StringTok{"Positive"}\NormalTok{,}
\NormalTok{  Outcome}\SpecialCharTok{==}\DecValTok{0} \SpecialCharTok{\textasciitilde{}} \StringTok{"Negative"}
\NormalTok{))}
\NormalTok{diabetes\_2}\SpecialCharTok{$}\NormalTok{Outcome }\OtherTok{\textless{}{-}} \FunctionTok{as.factor}\NormalTok{(diabetes\_2}\SpecialCharTok{$}\NormalTok{Outcome)}
\FunctionTok{summary}\NormalTok{(diabetes\_2)}
\end{Highlighting}
\end{Shaded}

\begin{verbatim}
##   Pregnancies        Glucose      BloodPressure    SkinThickness  
##  Min.   : 0.000   Min.   :  0.0   Min.   :  0.00   Min.   : 0.00  
##  1st Qu.: 1.000   1st Qu.: 99.0   1st Qu.: 62.00   1st Qu.: 0.00  
##  Median : 3.000   Median :117.0   Median : 72.00   Median :23.00  
##  Mean   : 3.845   Mean   :120.9   Mean   : 69.11   Mean   :20.54  
##  3rd Qu.: 6.000   3rd Qu.:140.2   3rd Qu.: 80.00   3rd Qu.:32.00  
##  Max.   :17.000   Max.   :199.0   Max.   :122.00   Max.   :99.00  
##     Insulin           BMI        DiabetesPedigreeFunction      Age       
##  Min.   :  0.0   Min.   : 0.00   Min.   :0.0780           Min.   :21.00  
##  1st Qu.:  0.0   1st Qu.:27.30   1st Qu.:0.2437           1st Qu.:24.00  
##  Median : 30.5   Median :32.00   Median :0.3725           Median :29.00  
##  Mean   : 79.8   Mean   :31.99   Mean   :0.4719           Mean   :33.24  
##  3rd Qu.:127.2   3rd Qu.:36.60   3rd Qu.:0.6262           3rd Qu.:41.00  
##  Max.   :846.0   Max.   :67.10   Max.   :2.4200           Max.   :81.00  
##      Outcome   
##  Negative:500  
##  Positive:268  
##                
##                
##                
## 
\end{verbatim}

\#here the numeric variable ``outcome'' is transposed into categorical
variable, any value greater than zero is assigned as positive and else
negative.

\hypertarget{plot}{%
\subsection{Plot}\label{plot}}

\begin{Shaded}
\begin{Highlighting}[]
\FunctionTok{plot}\NormalTok{(diabetes\_2}\SpecialCharTok{$}\NormalTok{Glucose)}
\end{Highlighting}
\end{Shaded}

\includegraphics{Assignment-1_files/figure-latex/plot-1.pdf}

\#it creates a scatter plot of variable ``glucose'' levels of the
patients, the max glucose level is 199.0 in the data set.

\end{document}
