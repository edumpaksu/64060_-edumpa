% Options for packages loaded elsewhere
\PassOptionsToPackage{unicode}{hyperref}
\PassOptionsToPackage{hyphens}{url}
%
\documentclass[
]{article}
\usepackage{amsmath,amssymb}
\usepackage{iftex}
\ifPDFTeX
  \usepackage[T1]{fontenc}
  \usepackage[utf8]{inputenc}
  \usepackage{textcomp} % provide euro and other symbols
\else % if luatex or xetex
  \usepackage{unicode-math} % this also loads fontspec
  \defaultfontfeatures{Scale=MatchLowercase}
  \defaultfontfeatures[\rmfamily]{Ligatures=TeX,Scale=1}
\fi
\usepackage{lmodern}
\ifPDFTeX\else
  % xetex/luatex font selection
\fi
% Use upquote if available, for straight quotes in verbatim environments
\IfFileExists{upquote.sty}{\usepackage{upquote}}{}
\IfFileExists{microtype.sty}{% use microtype if available
  \usepackage[]{microtype}
  \UseMicrotypeSet[protrusion]{basicmath} % disable protrusion for tt fonts
}{}
\makeatletter
\@ifundefined{KOMAClassName}{% if non-KOMA class
  \IfFileExists{parskip.sty}{%
    \usepackage{parskip}
  }{% else
    \setlength{\parindent}{0pt}
    \setlength{\parskip}{6pt plus 2pt minus 1pt}}
}{% if KOMA class
  \KOMAoptions{parskip=half}}
\makeatother
\usepackage{xcolor}
\usepackage[margin=1in]{geometry}
\usepackage{color}
\usepackage{fancyvrb}
\newcommand{\VerbBar}{|}
\newcommand{\VERB}{\Verb[commandchars=\\\{\}]}
\DefineVerbatimEnvironment{Highlighting}{Verbatim}{commandchars=\\\{\}}
% Add ',fontsize=\small' for more characters per line
\usepackage{framed}
\definecolor{shadecolor}{RGB}{248,248,248}
\newenvironment{Shaded}{\begin{snugshade}}{\end{snugshade}}
\newcommand{\AlertTok}[1]{\textcolor[rgb]{0.94,0.16,0.16}{#1}}
\newcommand{\AnnotationTok}[1]{\textcolor[rgb]{0.56,0.35,0.01}{\textbf{\textit{#1}}}}
\newcommand{\AttributeTok}[1]{\textcolor[rgb]{0.13,0.29,0.53}{#1}}
\newcommand{\BaseNTok}[1]{\textcolor[rgb]{0.00,0.00,0.81}{#1}}
\newcommand{\BuiltInTok}[1]{#1}
\newcommand{\CharTok}[1]{\textcolor[rgb]{0.31,0.60,0.02}{#1}}
\newcommand{\CommentTok}[1]{\textcolor[rgb]{0.56,0.35,0.01}{\textit{#1}}}
\newcommand{\CommentVarTok}[1]{\textcolor[rgb]{0.56,0.35,0.01}{\textbf{\textit{#1}}}}
\newcommand{\ConstantTok}[1]{\textcolor[rgb]{0.56,0.35,0.01}{#1}}
\newcommand{\ControlFlowTok}[1]{\textcolor[rgb]{0.13,0.29,0.53}{\textbf{#1}}}
\newcommand{\DataTypeTok}[1]{\textcolor[rgb]{0.13,0.29,0.53}{#1}}
\newcommand{\DecValTok}[1]{\textcolor[rgb]{0.00,0.00,0.81}{#1}}
\newcommand{\DocumentationTok}[1]{\textcolor[rgb]{0.56,0.35,0.01}{\textbf{\textit{#1}}}}
\newcommand{\ErrorTok}[1]{\textcolor[rgb]{0.64,0.00,0.00}{\textbf{#1}}}
\newcommand{\ExtensionTok}[1]{#1}
\newcommand{\FloatTok}[1]{\textcolor[rgb]{0.00,0.00,0.81}{#1}}
\newcommand{\FunctionTok}[1]{\textcolor[rgb]{0.13,0.29,0.53}{\textbf{#1}}}
\newcommand{\ImportTok}[1]{#1}
\newcommand{\InformationTok}[1]{\textcolor[rgb]{0.56,0.35,0.01}{\textbf{\textit{#1}}}}
\newcommand{\KeywordTok}[1]{\textcolor[rgb]{0.13,0.29,0.53}{\textbf{#1}}}
\newcommand{\NormalTok}[1]{#1}
\newcommand{\OperatorTok}[1]{\textcolor[rgb]{0.81,0.36,0.00}{\textbf{#1}}}
\newcommand{\OtherTok}[1]{\textcolor[rgb]{0.56,0.35,0.01}{#1}}
\newcommand{\PreprocessorTok}[1]{\textcolor[rgb]{0.56,0.35,0.01}{\textit{#1}}}
\newcommand{\RegionMarkerTok}[1]{#1}
\newcommand{\SpecialCharTok}[1]{\textcolor[rgb]{0.81,0.36,0.00}{\textbf{#1}}}
\newcommand{\SpecialStringTok}[1]{\textcolor[rgb]{0.31,0.60,0.02}{#1}}
\newcommand{\StringTok}[1]{\textcolor[rgb]{0.31,0.60,0.02}{#1}}
\newcommand{\VariableTok}[1]{\textcolor[rgb]{0.00,0.00,0.00}{#1}}
\newcommand{\VerbatimStringTok}[1]{\textcolor[rgb]{0.31,0.60,0.02}{#1}}
\newcommand{\WarningTok}[1]{\textcolor[rgb]{0.56,0.35,0.01}{\textbf{\textit{#1}}}}
\usepackage{graphicx}
\makeatletter
\def\maxwidth{\ifdim\Gin@nat@width>\linewidth\linewidth\else\Gin@nat@width\fi}
\def\maxheight{\ifdim\Gin@nat@height>\textheight\textheight\else\Gin@nat@height\fi}
\makeatother
% Scale images if necessary, so that they will not overflow the page
% margins by default, and it is still possible to overwrite the defaults
% using explicit options in \includegraphics[width, height, ...]{}
\setkeys{Gin}{width=\maxwidth,height=\maxheight,keepaspectratio}
% Set default figure placement to htbp
\makeatletter
\def\fps@figure{htbp}
\makeatother
\setlength{\emergencystretch}{3em} % prevent overfull lines
\providecommand{\tightlist}{%
  \setlength{\itemsep}{0pt}\setlength{\parskip}{0pt}}
\setcounter{secnumdepth}{-\maxdimen} % remove section numbering
\ifLuaTeX
  \usepackage{selnolig}  % disable illegal ligatures
\fi
\IfFileExists{bookmark.sty}{\usepackage{bookmark}}{\usepackage{hyperref}}
\IfFileExists{xurl.sty}{\usepackage{xurl}}{} % add URL line breaks if available
\urlstyle{same}
\hypersetup{
  pdftitle={ASSIGNMENT 2 FML ESWAR DUMPA},
  pdfauthor={Eswar Dumpa},
  hidelinks,
  pdfcreator={LaTeX via pandoc}}

\title{ASSIGNMENT 2 FML ESWAR DUMPA}
\author{Eswar Dumpa}
\date{2024-02-25}

\begin{document}
\maketitle

\#Overview

\#Questions - Responses

\begin{enumerate}
\def\labelenumi{\arabic{enumi}.}
\item
  How would this customer be classified? A. Since the new client does
  not take out a personal loan, they would be categorized as 0.
\item
  What is a choice of k that balances between overfitting and ignoring
  the predictor information? A. With an overall efficiency of 0, the
  optimal value of K is 3.
\item
  Show the confusion matrix for the validation data that results from
  using the best k. A. By using the best value for K as 3, and at
  set.seed(159) the confusion matrix was

\begin{verbatim}
   Reference
\end{verbatim}

  Prediction 0 1 0 1811 61 1 7 121

  True positive = 121 True Negative = 1811 False Positive = 7 False
  Negative = 61
\end{enumerate}

4.Classify the customer using the best k? A. Based on the best value of
K, which is K=3, the client would be categorized as 0. Thus, the client
declines the personal loan.

\begin{enumerate}
\def\labelenumi{\arabic{enumi}.}
\setcounter{enumi}{4}
\tightlist
\item
  Repartition the data, this time into training, validation, and test
  sets (50\% : 30\% : 20\%). Apply the k-NN method with the k chosen
  above. Compare the confusion matrix of the test set with that of the
  training and validation sets. Comment on the differences and their
  reason.
\end{enumerate}

A. Compared to test and validation data sets, the training set has
higher accuracy (97.4\%), sensitivity (75.93\%), and specificity
(99.7\%). It was caused by a number of things, including sample size,
data leaking, and overfitting. The primary cause was overfitting, which
occurs when a model is given permission to commit training data to
memory in order to capture all of the training data's headlines.
Consequently, the model's performance on training data will be
remarkably higher than that of the other two data sets.

For Testing data: Accuracy was 95.60\% Sensitivity was 60.64\%
Specificity was 99.23\%

For Validation data: Accuracy was 96.13\% Sensitivity was 65.51\%
Specificity was 99.41\%

For Training data: Accuracy was 97.44\% Sensitivity was 75.93\%
Specificity was 99.73\%

\hypertarget{loaded-the-required-libraries}{%
\section{loaded the required
libraries}\label{loaded-the-required-libraries}}

\begin{Shaded}
\begin{Highlighting}[]
\FunctionTok{library}\NormalTok{(class)}
\FunctionTok{library}\NormalTok{(caret)}
\end{Highlighting}
\end{Shaded}

\begin{verbatim}
## Loading required package: ggplot2
\end{verbatim}

\begin{verbatim}
## Loading required package: lattice
\end{verbatim}

\begin{Shaded}
\begin{Highlighting}[]
\FunctionTok{library}\NormalTok{(e1071)}
\FunctionTok{library}\NormalTok{(ggplot2)}
\FunctionTok{library}\NormalTok{(lattice)}
\end{Highlighting}
\end{Shaded}

\begin{Shaded}
\begin{Highlighting}[]
\CommentTok{\#Data import }
\NormalTok{universal\_bank }\OtherTok{\textless{}{-}} \FunctionTok{read.csv}\NormalTok{(}\StringTok{"C:/Users/eshwa/Documents/Fundamentals of Machine Learning/Assignment 2/UniversalBank.csv"}\NormalTok{)}
\FunctionTok{dim}\NormalTok{(universal\_bank)}
\end{Highlighting}
\end{Shaded}

\begin{verbatim}
## [1] 5000   14
\end{verbatim}

\begin{Shaded}
\begin{Highlighting}[]
\CommentTok{\# t function creates the transpose of the dataframe}
\FunctionTok{t}\NormalTok{(}\FunctionTok{t}\NormalTok{(}\FunctionTok{names}\NormalTok{(universal\_bank)))}
\end{Highlighting}
\end{Shaded}

\begin{verbatim}
##       [,1]                
##  [1,] "ID"                
##  [2,] "Age"               
##  [3,] "Experience"        
##  [4,] "Income"            
##  [5,] "ZIP.Code"          
##  [6,] "Family"            
##  [7,] "CCAvg"             
##  [8,] "Education"         
##  [9,] "Mortgage"          
## [10,] "Personal.Loan"     
## [11,] "Securities.Account"
## [12,] "CD.Account"        
## [13,] "Online"            
## [14,] "CreditCard"
\end{verbatim}

\#PUT ID AND ZIP

\begin{Shaded}
\begin{Highlighting}[]
\CommentTok{\#here 1 and 5 are the indexes for the columns ID and ZIP}
\NormalTok{universal\_bank }\OtherTok{\textless{}{-}}\NormalTok{ universal\_bank[,}\SpecialCharTok{{-}}\FunctionTok{c}\NormalTok{(}\DecValTok{1}\NormalTok{,}\DecValTok{5}\NormalTok{)] }
\FunctionTok{dim}\NormalTok{(universal\_bank)}
\end{Highlighting}
\end{Shaded}

\begin{verbatim}
## [1] 5000   12
\end{verbatim}

\begin{Shaded}
\begin{Highlighting}[]
\CommentTok{\#education only need to be converted into factor}
\NormalTok{universal\_bank}\SpecialCharTok{$}\NormalTok{Education }\OtherTok{\textless{}{-}} \FunctionTok{as.factor}\NormalTok{(universal\_bank}\SpecialCharTok{$}\NormalTok{Education)}

\CommentTok{\# converting education level to dummy variables}
\NormalTok{groups }\OtherTok{\textless{}{-}} \FunctionTok{dummyVars}\NormalTok{(}\SpecialCharTok{\textasciitilde{}}\NormalTok{.,}\AttributeTok{data=}\NormalTok{universal\_bank)}
\NormalTok{universal\_B\_bank }\OtherTok{\textless{}{-}} \FunctionTok{as.data.frame}\NormalTok{(}\FunctionTok{predict}\NormalTok{(groups,universal\_bank))}
\end{Highlighting}
\end{Shaded}

\begin{Shaded}
\begin{Highlighting}[]
\CommentTok{\#gives us same sample if we return the code}
\FunctionTok{set.seed}\NormalTok{(}\DecValTok{159}\NormalTok{) }

\CommentTok{\#60\% training data}
\NormalTok{train\_index }\OtherTok{\textless{}{-}} \FunctionTok{sample}\NormalTok{(}\FunctionTok{row.names}\NormalTok{(universal\_B\_bank), }\FloatTok{0.6}\SpecialCharTok{*}\FunctionTok{dim}\NormalTok{(universal\_B\_bank)[}\DecValTok{1}\NormalTok{])}
\NormalTok{train\_bank }\OtherTok{\textless{}{-}}\NormalTok{ universal\_B\_bank[train\_index,]}

\CommentTok{\#40\% validation data}
\NormalTok{valid\_index }\OtherTok{\textless{}{-}} \FunctionTok{setdiff}\NormalTok{(}\FunctionTok{row.names}\NormalTok{(universal\_B\_bank), train\_index)}
\NormalTok{valid\_bank }\OtherTok{\textless{}{-}}\NormalTok{ universal\_B\_bank[valid\_index,]}

\CommentTok{\# Prints the dims of the datasets}
\FunctionTok{cat}\NormalTok{(}\StringTok{"Training data dimensions:"}\NormalTok{, }\FunctionTok{dim}\NormalTok{(train\_bank), }\StringTok{"}\SpecialCharTok{\textbackslash{}n}\StringTok{"}\NormalTok{)}
\end{Highlighting}
\end{Shaded}

\begin{verbatim}
## Training data dimensions: 3000 14
\end{verbatim}

\begin{Shaded}
\begin{Highlighting}[]
\FunctionTok{cat}\NormalTok{(}\StringTok{"Validation data dimensions:"}\NormalTok{, }\FunctionTok{dim}\NormalTok{(valid\_bank), }\StringTok{"}\SpecialCharTok{\textbackslash{}n}\StringTok{"}\NormalTok{)}
\end{Highlighting}
\end{Shaded}

\begin{verbatim}
## Validation data dimensions: 2000 14
\end{verbatim}

\begin{Shaded}
\begin{Highlighting}[]
\CommentTok{\# 10th variable of the data frame is personal loan}
\NormalTok{train\_norm\_bank }\OtherTok{\textless{}{-}}\NormalTok{ train\_bank[,}\SpecialCharTok{{-}}\DecValTok{10}\NormalTok{] }\CommentTok{\# Personal loan is the 10th variable in data frame}
\NormalTok{valid\_norm\_bank }\OtherTok{\textless{}{-}}\NormalTok{ valid\_bank[,}\SpecialCharTok{{-}}\DecValTok{10}\NormalTok{]}

\NormalTok{norm.values }\OtherTok{\textless{}{-}} \FunctionTok{preProcess}\NormalTok{(train\_bank[, }\SpecialCharTok{{-}}\DecValTok{10}\NormalTok{], }\AttributeTok{method=}\FunctionTok{c}\NormalTok{(}\StringTok{"center"}\NormalTok{, }\StringTok{"scale"}\NormalTok{))}

\CommentTok{\#Normalization of training dataset and validation dataset}
\NormalTok{train\_norm\_bank }\OtherTok{\textless{}{-}} \FunctionTok{predict}\NormalTok{(norm.values, train\_bank[, }\SpecialCharTok{{-}}\DecValTok{10}\NormalTok{])}
\NormalTok{valid\_norm\_bank }\OtherTok{\textless{}{-}} \FunctionTok{predict}\NormalTok{(norm.values, valid\_bank[, }\SpecialCharTok{{-}}\DecValTok{10}\NormalTok{])}
\end{Highlighting}
\end{Shaded}

\begin{enumerate}
\def\labelenumi{\arabic{enumi})}
\tightlist
\item
  Age = 40, Experience = 10, Income = 84, Family = 2, CCAvg = 2,
  Education\_1 = 0, Education\_2 = 1, Education\_3 = 0, Mortgage = 0,
  Securities Account = 0, CD Account = 0, Online = 1, and Credit Card =
  1. Perform a k-NN classification with all predictors except ID and ZIP
  code using k = 1. Remember to transform categorical predictors with
  more than two categories into dummy variables first. Specify the
  success class as 1 (loan acceptance), and use the default cutoff value
  of 0.5. How would this customer be classified?
\end{enumerate}

\begin{Shaded}
\begin{Highlighting}[]
\CommentTok{\#creating a new customer input}
\NormalTok{new\_customer }\OtherTok{\textless{}{-}} \FunctionTok{data.frame}\NormalTok{(}
  \AttributeTok{Age =} \DecValTok{40}\NormalTok{,}
  \AttributeTok{Experience =} \DecValTok{10}\NormalTok{,}
  \AttributeTok{Income =} \DecValTok{84}\NormalTok{,}
  \AttributeTok{Family =} \DecValTok{2}\NormalTok{,}
  \AttributeTok{CCAvg =} \DecValTok{2}\NormalTok{,}
  \AttributeTok{Education.1 =} \DecValTok{0}\NormalTok{,}
  \AttributeTok{Education.2 =} \DecValTok{1}\NormalTok{,}
  \AttributeTok{Education.3 =} \DecValTok{0}\NormalTok{,}
  \AttributeTok{Mortgage =} \DecValTok{0}\NormalTok{,}
  \AttributeTok{Securities.Account =} \DecValTok{0}\NormalTok{,}
  \AttributeTok{CD.Account =} \DecValTok{0}\NormalTok{,}
  \AttributeTok{Online =} \DecValTok{1}\NormalTok{,}
  \AttributeTok{CreditCard =} \DecValTok{1}
\NormalTok{)}

\CommentTok{\# Normalizing}
\NormalTok{new\_cust\_norm }\OtherTok{\textless{}{-}}\NormalTok{ new\_customer}
\NormalTok{new\_cust\_norm }\OtherTok{\textless{}{-}} \FunctionTok{predict}\NormalTok{(norm.values, new\_cust\_norm)}
\end{Highlighting}
\end{Shaded}

\begin{Shaded}
\begin{Highlighting}[]
\CommentTok{\#Assuming k=1}
\NormalTok{knn.pred1 }\OtherTok{\textless{}{-}}\NormalTok{ class}\SpecialCharTok{::}\FunctionTok{knn}\NormalTok{(}\AttributeTok{train =}\NormalTok{ train\_norm\_bank, }
                       \AttributeTok{test =}\NormalTok{ new\_cust\_norm, }
                       \AttributeTok{cl =}\NormalTok{ train\_bank}\SpecialCharTok{$}\NormalTok{Personal.Loan, }\AttributeTok{k =} \DecValTok{1}\NormalTok{)}

\CommentTok{\# Prints knn prediction}
\NormalTok{knn.pred1}
\end{Highlighting}
\end{Shaded}

\begin{verbatim}
## [1] 0
## Levels: 0 1
\end{verbatim}

\#loan not granted for given test dataset for k=1

\begin{enumerate}
\def\labelenumi{\arabic{enumi})}
\setcounter{enumi}{1}
\tightlist
\item
  What is a choice of k that balances between over fitting and ignoring
  the predictor information?
\end{enumerate}

\begin{Shaded}
\begin{Highlighting}[]
\CommentTok{\#set range of k 1 to 20}
\NormalTok{accuracy\_bank }\OtherTok{\textless{}{-}} \FunctionTok{data.frame}\NormalTok{(}\AttributeTok{k =} \FunctionTok{seq}\NormalTok{(}\DecValTok{1}\NormalTok{, }\DecValTok{20}\NormalTok{, }\DecValTok{1}\NormalTok{), }\AttributeTok{overallaccuracy =} \FunctionTok{rep}\NormalTok{(}\DecValTok{0}\NormalTok{, }\DecValTok{20}\NormalTok{))}

\ControlFlowTok{for}\NormalTok{(i }\ControlFlowTok{in} \DecValTok{1}\SpecialCharTok{:}\DecValTok{20}\NormalTok{) \{}
\NormalTok{knn.pred }\OtherTok{\textless{}{-}}\NormalTok{ class}\SpecialCharTok{::}\FunctionTok{knn}\NormalTok{(}\AttributeTok{train =}\NormalTok{ train\_norm\_bank, }
                       \AttributeTok{test =}\NormalTok{ valid\_norm\_bank, }
                       \AttributeTok{cl =}\NormalTok{ train\_bank}\SpecialCharTok{$}\NormalTok{Personal.Loan, }\AttributeTok{k =}\NormalTok{ i)}


\NormalTok{accuracy\_bank[i, }\DecValTok{2}\NormalTok{] }\OtherTok{\textless{}{-}} \FunctionTok{confusionMatrix}\NormalTok{(knn.pred, }\FunctionTok{as.factor}\NormalTok{(valid\_bank}\SpecialCharTok{$}\NormalTok{Personal.Loan),}
                                       \AttributeTok{positive =} \StringTok{"1"}\NormalTok{)}\SpecialCharTok{$}\NormalTok{overall[}\DecValTok{1}\NormalTok{] }
\NormalTok{\}}

\CommentTok{\#k value with max accuracy}
\NormalTok{bestValueofk }\OtherTok{\textless{}{-}} \FunctionTok{which}\NormalTok{(accuracy\_bank[,}\DecValTok{2}\NormalTok{] }\SpecialCharTok{==} \FunctionTok{max}\NormalTok{(accuracy\_bank[,}\DecValTok{2}\NormalTok{])) }\CommentTok{\# gives the k value with maximum accuracy}
\NormalTok{accuracy\_bank}
\end{Highlighting}
\end{Shaded}

\begin{verbatim}
##     k overallaccuracy
## 1   1          0.9620
## 2   2          0.9630
## 3   3          0.9660
## 4   4          0.9615
## 5   5          0.9630
## 6   6          0.9600
## 7   7          0.9630
## 8   8          0.9630
## 9   9          0.9600
## 10 10          0.9575
## 11 11          0.9570
## 12 12          0.9540
## 13 13          0.9535
## 14 14          0.9525
## 15 15          0.9510
## 16 16          0.9500
## 17 17          0.9495
## 18 18          0.9465
## 19 19          0.9465
## 20 20          0.9460
\end{verbatim}

\begin{Shaded}
\begin{Highlighting}[]
\CommentTok{\#prints the best value of k}
\FunctionTok{cat}\NormalTok{(}\StringTok{"The Best Value of k is:"}\NormalTok{, bestValueofk)}
\end{Highlighting}
\end{Shaded}

\begin{verbatim}
## The Best Value of k is: 3
\end{verbatim}

\begin{Shaded}
\begin{Highlighting}[]
\CommentTok{\# The Best Value of k is 3}
\CommentTok{\#Plotting graph between k value and accuracy}
\FunctionTok{plot}\NormalTok{(accuracy\_bank}\SpecialCharTok{$}\NormalTok{k,accuracy\_bank}\SpecialCharTok{$}\NormalTok{overallaccuracy)}
\end{Highlighting}
\end{Shaded}

\includegraphics{Assignment-2_files/figure-latex/unnamed-chunk-10-1.pdf}

\begin{enumerate}
\def\labelenumi{\arabic{enumi})}
\setcounter{enumi}{2}
\tightlist
\item
  Show the confusion matrix for the validation data that results from
  using the best k.
\end{enumerate}

\begin{Shaded}
\begin{Highlighting}[]
\CommentTok{\# take the best k value for prediction}
\NormalTok{knn.pred2 }\OtherTok{\textless{}{-}}\NormalTok{ class}\SpecialCharTok{::}\FunctionTok{knn}\NormalTok{(}\AttributeTok{train =}\NormalTok{ train\_norm\_bank, }
                        \AttributeTok{test =}\NormalTok{ valid\_norm\_bank, }
                        \AttributeTok{cl =}\NormalTok{ train\_bank}\SpecialCharTok{$}\NormalTok{Personal.Loan, }\AttributeTok{k =}\NormalTok{ bestValueofk)}


\CommentTok{\# confusion matrix for dataset}
\NormalTok{confusion\_matrix }\OtherTok{\textless{}{-}} \FunctionTok{confusionMatrix}\NormalTok{(knn.pred2,}
                                    \FunctionTok{as.factor}\NormalTok{(valid\_bank}\SpecialCharTok{$}\NormalTok{Personal.Loan), }\AttributeTok{positive =} \StringTok{"1"}\NormalTok{)}

\FunctionTok{cat}\NormalTok{(}\StringTok{"Confusion Matrix for validation data:"}\NormalTok{, }\StringTok{"}\SpecialCharTok{\textbackslash{}n}\StringTok{"}\NormalTok{)}
\end{Highlighting}
\end{Shaded}

\begin{verbatim}
## Confusion Matrix for validation data:
\end{verbatim}

\begin{Shaded}
\begin{Highlighting}[]
\CommentTok{\# Confusion Matrix of validation data}
\FunctionTok{print}\NormalTok{(confusion\_matrix)}
\end{Highlighting}
\end{Shaded}

\begin{verbatim}
## Confusion Matrix and Statistics
## 
##           Reference
## Prediction    0    1
##          0 1811   61
##          1    7  121
##                                           
##                Accuracy : 0.966           
##                  95% CI : (0.9571, 0.9735)
##     No Information Rate : 0.909           
##     P-Value [Acc > NIR] : < 2e-16         
##                                           
##                   Kappa : 0.7628          
##                                           
##  Mcnemar's Test P-Value : 1.3e-10         
##                                           
##             Sensitivity : 0.6648          
##             Specificity : 0.9961          
##          Pos Pred Value : 0.9453          
##          Neg Pred Value : 0.9674          
##              Prevalence : 0.0910          
##          Detection Rate : 0.0605          
##    Detection Prevalence : 0.0640          
##       Balanced Accuracy : 0.8305          
##                                           
##        'Positive' Class : 1               
## 
\end{verbatim}

\begin{enumerate}
\def\labelenumi{\arabic{enumi}.}
\setcounter{enumi}{3}
\tightlist
\item
  Consider the following customer: Age = 40, Experience = 10, Income =
  84, Family = 2, CCAvg = 2, Education\_1 = 0, Education\_2 = 1,
  Education\_3 = 0, Mortgage = 0, Securities Account = 0, CD Account =
  0, Online = 1 and Credit Card = 1. Classify the customer using the
  best k.
\end{enumerate}

\begin{Shaded}
\begin{Highlighting}[]
\NormalTok{new\_customer1 }\OtherTok{\textless{}{-}} \FunctionTok{data.frame}\NormalTok{(}
  \AttributeTok{Age =} \DecValTok{40}\NormalTok{,}
  \AttributeTok{Experience =} \DecValTok{10}\NormalTok{,}
  \AttributeTok{Income =} \DecValTok{84}\NormalTok{,}
  \AttributeTok{Family =} \DecValTok{2}\NormalTok{,}
  \AttributeTok{CCAvg =} \DecValTok{2}\NormalTok{,}
  \AttributeTok{Education.1 =} \DecValTok{0}\NormalTok{,}
  \AttributeTok{Education.2 =} \DecValTok{1}\NormalTok{,}
  \AttributeTok{Education.3 =} \DecValTok{0}\NormalTok{,}
  \AttributeTok{Mortgage =} \DecValTok{0}\NormalTok{,}
  \AttributeTok{Securities.Account =} \DecValTok{0}\NormalTok{,}
  \AttributeTok{CD.Account =} \DecValTok{0}\NormalTok{,}
  \AttributeTok{Online =} \DecValTok{1}\NormalTok{,}
  \AttributeTok{CreditCard =} \DecValTok{1}
\NormalTok{)}

\CommentTok{\# Normalizing new customer}
\NormalTok{new\_cust\_norm1 }\OtherTok{\textless{}{-}}\NormalTok{ new\_customer1}
\NormalTok{new\_cust\_norm1 }\OtherTok{\textless{}{-}} \FunctionTok{predict}\NormalTok{(norm.values, new\_cust\_norm1)}
\end{Highlighting}
\end{Shaded}

\begin{Shaded}
\begin{Highlighting}[]
\NormalTok{knn.pred3 }\OtherTok{\textless{}{-}}\NormalTok{ class}\SpecialCharTok{::}\FunctionTok{knn}\NormalTok{(}\AttributeTok{train =}\NormalTok{ train\_norm\_bank, }
                        \AttributeTok{test =}\NormalTok{ new\_cust\_norm1, }
                        \AttributeTok{cl =}\NormalTok{ train\_bank}\SpecialCharTok{$}\NormalTok{Personal.Loan, }\AttributeTok{k =}\NormalTok{ bestValueofk)}

\CommentTok{\#prints prediction}
\NormalTok{knn.pred3}
\end{Highlighting}
\end{Shaded}

\begin{verbatim}
## [1] 0
## Levels: 0 1
\end{verbatim}

\begin{enumerate}
\def\labelenumi{\arabic{enumi}.}
\setcounter{enumi}{4}
\tightlist
\item
  Repartition the data, this time into training, validation, and test
  sets (50\% : 30\% : 20\%). Apply the k-NN method with the k chosen
  above. Compare the confusion matrix of the test set with that of the
  training and validation sets. Comment on the differences and their
  reason.
\end{enumerate}

\begin{Shaded}
\begin{Highlighting}[]
\FunctionTok{set.seed}\NormalTok{(}\DecValTok{159}\NormalTok{) }\CommentTok{\# Ensures that we get the same sample if we rerun the code}

\CommentTok{\# Split the data to training (50\%), validation (30\%) and testing (20\%) sets each}
\NormalTok{train\_index1 }\OtherTok{\textless{}{-}} \FunctionTok{sample}\NormalTok{(}\FunctionTok{row.names}\NormalTok{(universal\_B\_bank), }\FloatTok{0.5}\SpecialCharTok{*}\FunctionTok{dim}\NormalTok{(universal\_B\_bank)[}\DecValTok{1}\NormalTok{])}
\NormalTok{valid\_index1 }\OtherTok{\textless{}{-}} \FunctionTok{sample}\NormalTok{(}\FunctionTok{setdiff}\NormalTok{(}\FunctionTok{row.names}\NormalTok{(universal\_B\_bank), train\_index1),}
                       \FloatTok{0.3}\SpecialCharTok{*}\FunctionTok{dim}\NormalTok{(universal\_B\_bank)[}\DecValTok{1}\NormalTok{]) }
\NormalTok{test\_index1 }\OtherTok{\textless{}{-}} \FunctionTok{setdiff}\NormalTok{(}\FunctionTok{row.names}\NormalTok{(universal\_B\_bank), }\FunctionTok{c}\NormalTok{(train\_index1,valid\_index1))}

\NormalTok{train\_Data1 }\OtherTok{\textless{}{-}}\NormalTok{ universal\_B\_bank[train\_index1,]}
\NormalTok{valid\_Data1 }\OtherTok{\textless{}{-}}\NormalTok{ universal\_B\_bank[valid\_index1,]}
\NormalTok{test\_Data1 }\OtherTok{\textless{}{-}}\NormalTok{ universal\_B\_bank[test\_index1,]}

\CommentTok{\# Print dimensions of split datasets}
\FunctionTok{cat}\NormalTok{(}\StringTok{"Training data dimensions:"}\NormalTok{, }\FunctionTok{dim}\NormalTok{(train\_Data1), }\StringTok{"}\SpecialCharTok{\textbackslash{}n}\StringTok{"}\NormalTok{)}
\end{Highlighting}
\end{Shaded}

\begin{verbatim}
## Training data dimensions: 2500 14
\end{verbatim}

\begin{Shaded}
\begin{Highlighting}[]
\FunctionTok{cat}\NormalTok{(}\StringTok{"Validation data dimensions:"}\NormalTok{, }\FunctionTok{dim}\NormalTok{(valid\_Data1), }\StringTok{"}\SpecialCharTok{\textbackslash{}n}\StringTok{"}\NormalTok{)}
\end{Highlighting}
\end{Shaded}

\begin{verbatim}
## Validation data dimensions: 1500 14
\end{verbatim}

\begin{Shaded}
\begin{Highlighting}[]
\FunctionTok{cat}\NormalTok{(}\StringTok{"Testing data dimensions:"}\NormalTok{, }\FunctionTok{dim}\NormalTok{(test\_Data1), }\StringTok{"}\SpecialCharTok{\textbackslash{}n}\StringTok{"}\NormalTok{)}
\end{Highlighting}
\end{Shaded}

\begin{verbatim}
## Testing data dimensions: 1000 14
\end{verbatim}

\begin{Shaded}
\begin{Highlighting}[]
\CommentTok{\#Normalize data for 3 sets}
\NormalTok{train\_norm\_bank1 }\OtherTok{\textless{}{-}}\NormalTok{ train\_Data1[ ,}\SpecialCharTok{{-}}\DecValTok{10}\NormalTok{] }\CommentTok{\#removing the 10th variable(personal loan)}
\NormalTok{valid\_norm\_bank1 }\OtherTok{\textless{}{-}}\NormalTok{ valid\_Data1[ ,}\SpecialCharTok{{-}}\DecValTok{10}\NormalTok{]}
\NormalTok{test\_norm\_bank1 }\OtherTok{\textless{}{-}}\NormalTok{ test\_Data1[ ,}\SpecialCharTok{{-}}\DecValTok{10}\NormalTok{]}

\CommentTok{\#Preprocessing}
\NormalTok{norm.values1 }\OtherTok{\textless{}{-}} \FunctionTok{preProcess}\NormalTok{(train\_Data1[ ,}\SpecialCharTok{{-}}\DecValTok{10}\NormalTok{], }\AttributeTok{method=}\FunctionTok{c}\NormalTok{(}\StringTok{"center"}\NormalTok{, }\StringTok{"scale"}\NormalTok{))}
\NormalTok{train\_norm\_bank1 }\OtherTok{\textless{}{-}} \FunctionTok{predict}\NormalTok{(norm.values1, train\_Data1[ ,}\SpecialCharTok{{-}}\DecValTok{10}\NormalTok{])}
\NormalTok{valid\_norm\_bank1 }\OtherTok{\textless{}{-}} \FunctionTok{predict}\NormalTok{(norm.values1, valid\_Data1[ ,}\SpecialCharTok{{-}}\DecValTok{10}\NormalTok{])}
\NormalTok{test\_norm\_bank1 }\OtherTok{\textless{}{-}} \FunctionTok{predict}\NormalTok{(norm.values1, test\_Data1[ ,}\SpecialCharTok{{-}}\DecValTok{10}\NormalTok{])}
\end{Highlighting}
\end{Shaded}

\begin{Shaded}
\begin{Highlighting}[]
\CommentTok{\#knn prediction for best value of k}
\NormalTok{knn.pred.train }\OtherTok{\textless{}{-}}\NormalTok{ class}\SpecialCharTok{::}\FunctionTok{knn}\NormalTok{(}\AttributeTok{train =}\NormalTok{ train\_norm\_bank1, }
                             \AttributeTok{test =}\NormalTok{ train\_norm\_bank1, }
                             \AttributeTok{cl =}\NormalTok{ train\_Data1}\SpecialCharTok{$}\NormalTok{Personal.Loan, }\AttributeTok{k =} \DecValTok{3}\NormalTok{)}

\CommentTok{\#confusion matrix of training data}
\NormalTok{confusion\_matrix.train }\OtherTok{\textless{}{-}} \FunctionTok{confusionMatrix}\NormalTok{(knn.pred.train, }
                                          \FunctionTok{as.factor}\NormalTok{(train\_Data1}\SpecialCharTok{$}\NormalTok{Personal.Loan), }\AttributeTok{positive =} \StringTok{"1"}\NormalTok{)}

\CommentTok{\#print matrix}
\FunctionTok{cat}\NormalTok{(}\StringTok{"Confusion Matrix for training data:"}\NormalTok{, }\StringTok{"}\SpecialCharTok{\textbackslash{}n}\StringTok{"}\NormalTok{)}
\end{Highlighting}
\end{Shaded}

\begin{verbatim}
## Confusion Matrix for training data:
\end{verbatim}

\begin{Shaded}
\begin{Highlighting}[]
\CommentTok{\#Confusion Matrix of training data:}
\FunctionTok{print}\NormalTok{(confusion\_matrix.train)}
\end{Highlighting}
\end{Shaded}

\begin{verbatim}
## Confusion Matrix and Statistics
## 
##           Reference
## Prediction    0    1
##          0 2253   58
##          1    6  183
##                                           
##                Accuracy : 0.9744          
##                  95% CI : (0.9674, 0.9802)
##     No Information Rate : 0.9036          
##     P-Value [Acc > NIR] : < 2.2e-16       
##                                           
##                   Kappa : 0.8374          
##                                           
##  Mcnemar's Test P-Value : 1.83e-10        
##                                           
##             Sensitivity : 0.7593          
##             Specificity : 0.9973          
##          Pos Pred Value : 0.9683          
##          Neg Pred Value : 0.9749          
##              Prevalence : 0.0964          
##          Detection Rate : 0.0732          
##    Detection Prevalence : 0.0756          
##       Balanced Accuracy : 0.8783          
##                                           
##        'Positive' Class : 1               
## 
\end{verbatim}

\begin{Shaded}
\begin{Highlighting}[]
\NormalTok{knn.pred.valid }\OtherTok{\textless{}{-}}\NormalTok{ class}\SpecialCharTok{::}\FunctionTok{knn}\NormalTok{(}\AttributeTok{train =}\NormalTok{ train\_norm\_bank1, }
                             \AttributeTok{test =}\NormalTok{ valid\_norm\_bank1, }
                             \AttributeTok{cl =}\NormalTok{ train\_Data1}\SpecialCharTok{$}\NormalTok{Personal.Loan, }\AttributeTok{k =}\NormalTok{ bestValueofk)}

\CommentTok{\#confusion matrix }
\NormalTok{confusion\_matrix.valid }\OtherTok{\textless{}{-}} \FunctionTok{confusionMatrix}\NormalTok{(knn.pred.valid, }
                                          \FunctionTok{as.factor}\NormalTok{(valid\_Data1}\SpecialCharTok{$}\NormalTok{Personal.Loan), }\AttributeTok{positive =} \StringTok{"1"}\NormalTok{)}

\CommentTok{\#print matrix}
\FunctionTok{cat}\NormalTok{(}\StringTok{"Confusion Matrix for Validation data:"}\NormalTok{, }\StringTok{"}\SpecialCharTok{\textbackslash{}n}\StringTok{"}\NormalTok{)}
\end{Highlighting}
\end{Shaded}

\begin{verbatim}
## Confusion Matrix for Validation data:
\end{verbatim}

\begin{Shaded}
\begin{Highlighting}[]
\CommentTok{\#Confusion Matrix }
\FunctionTok{print}\NormalTok{(confusion\_matrix.valid)}
\end{Highlighting}
\end{Shaded}

\begin{verbatim}
## Confusion Matrix and Statistics
## 
##           Reference
## Prediction    0    1
##          0 1347   50
##          1    8   95
##                                           
##                Accuracy : 0.9613          
##                  95% CI : (0.9503, 0.9705)
##     No Information Rate : 0.9033          
##     P-Value [Acc > NIR] : < 2.2e-16       
##                                           
##                   Kappa : 0.7457          
##                                           
##  Mcnemar's Test P-Value : 7.303e-08       
##                                           
##             Sensitivity : 0.65517         
##             Specificity : 0.99410         
##          Pos Pred Value : 0.92233         
##          Neg Pred Value : 0.96421         
##              Prevalence : 0.09667         
##          Detection Rate : 0.06333         
##    Detection Prevalence : 0.06867         
##       Balanced Accuracy : 0.82463         
##                                           
##        'Positive' Class : 1               
## 
\end{verbatim}

\begin{Shaded}
\begin{Highlighting}[]
\CommentTok{\#knn prediction for best value of k}
\NormalTok{knn.pred.test }\OtherTok{\textless{}{-}}\NormalTok{ class}\SpecialCharTok{::}\FunctionTok{knn}\NormalTok{(}\AttributeTok{train =}\NormalTok{ train\_norm\_bank1, }
                            \AttributeTok{test =}\NormalTok{ test\_norm\_bank1, }
                            \AttributeTok{cl =}\NormalTok{ train\_Data1}\SpecialCharTok{$}\NormalTok{Personal.Loan, }\AttributeTok{k =}\NormalTok{ bestValueofk)}

\CommentTok{\#confusion matrix }
\NormalTok{confusion\_matrix.test }\OtherTok{\textless{}{-}} \FunctionTok{confusionMatrix}\NormalTok{(knn.pred.test, }
                                        \FunctionTok{as.factor}\NormalTok{(test\_Data1}\SpecialCharTok{$}\NormalTok{Personal.Loan), }\AttributeTok{positive =} \StringTok{"1"}\NormalTok{)}

\CommentTok{\#print matrix}
\FunctionTok{cat}\NormalTok{(}\StringTok{"Confusion Matrix for Test data:"}\NormalTok{, }\StringTok{"}\SpecialCharTok{\textbackslash{}n}\StringTok{"}\NormalTok{)}
\end{Highlighting}
\end{Shaded}

\begin{verbatim}
## Confusion Matrix for Test data:
\end{verbatim}

\begin{Shaded}
\begin{Highlighting}[]
\CommentTok{\# Confusion Matrix }
\FunctionTok{print}\NormalTok{(confusion\_matrix.test)}
\end{Highlighting}
\end{Shaded}

\begin{verbatim}
## Confusion Matrix and Statistics
## 
##           Reference
## Prediction   0   1
##          0 899  37
##          1   7  57
##                                           
##                Accuracy : 0.956           
##                  95% CI : (0.9414, 0.9679)
##     No Information Rate : 0.906           
##     P-Value [Acc > NIR] : 1.733e-09       
##                                           
##                   Kappa : 0.6986          
##                                           
##  Mcnemar's Test P-Value : 1.232e-05       
##                                           
##             Sensitivity : 0.6064          
##             Specificity : 0.9923          
##          Pos Pred Value : 0.8906          
##          Neg Pred Value : 0.9605          
##              Prevalence : 0.0940          
##          Detection Rate : 0.0570          
##    Detection Prevalence : 0.0640          
##       Balanced Accuracy : 0.7993          
##                                           
##        'Positive' Class : 1               
## 
\end{verbatim}

\end{document}
